%!TEX root = THinstituteReport_1.tex

%%%%%%%%%%%%%%%%%%%%%%%%%%%%%%%%%%%%%%%%%%%%
\section{Introduction}
\label{sec:intro}
%%%%%%%%%%%%%%%%%%%%%%%%%%%%%%%%%%%%%%%%%%%%

%In recent years, observables involving fully reconstructed jets, produced in heavy-ion collisions, have been established as important tools to probe the properties of the underlying hot and dense QCD medium. From a reciprocal point of view, they also shed light on the details of parton fragmentation and hadronisation in the presence of final-state interactions. These developments are mainly driven by a dedicated experimental effort that aim at fully exploit the detector capabilities and experimental techniques at current high-energy colliders. On the theory side, an enhanced level of sophistication of the modeling of both jet fragmentation and its coupling to the medium evolution, often implemented as Monte-Carlo parton showers or event generators, allow to scrutinize the details of jet-medium interactions using a wide range of high-$\pT$ processes.
%
%Jets are excellent probes of the quark-gluon plasma (QGP) for several reasons; here we briefly outline but a few. Firstly, since they are multi-parton objects containing color degrees of freedom they are affected by final-state interactions with a deconfined medium. Secondly, in course of their evolution jets can probe various length scales of the QGP, providing a complementary window with regard to bulk observables that are dominated by soft physics.
%
%The way in which high-$\pT$ observables in heavy-ion collisions deviate from baseline measurements in proton-proton collisions is generically referred to as ``jet quenching''. This term arose historically in connection with the  first data on the suppression on high-$\pT$ hadron suppression and related di-jet/photon-jet asymmetries measured at RHIC. These processes clearly involve a large variety of scales, ranging from the jet scale to thermal scales of the QGP, as will also be explained in more detail below. Typical jet quenching observables will therefore be sensitive to perturbative and non-perturbative contributions in varying degree. This can complicate the interpretation of the considered observables and obfuscate the extraction of information about the underlying medium. Besides, additional background subtraction algorithms have to be applied in the more noisy heavy-ion environment, complicating the procedure of statistically removing detector effects (unfolding). Therefore a joint community effort between theory and experiment is crucial to fully take advantage of the potential of these QGP probes. 
%
%Many of these complications are less pressing or absent when measuring single-hadron spectra or hadron correlations. Hence, one notable example of such an joint effort was the so-called ``brick problem'' \cite{Armesto:2011ht,Burke:2013yra}, organized by the TECHQM collaboration \cite{TECHQM}, which aimed at clarifying differences and similarities between various theoretical implementations of jet quenching on the level of single-hadron spectra in heavy-ion collisions. The simplification allowing for such a comprehensive comparison was to reduce the complexity of the modeling of the underlying medium to a static, one-dimensional ``brick'' with constant density. 
%%While the employed models gave, as a result of comparison with the data, a wide spread of the obtained fundamental parameters, 
%Based on this exercise, it resulted in a comprehensive effort to estimate the jet quenching parameter $\hat q$ along with its uncertainties \cite{Burke:2013yra}.
%
%The joint organization of the 5th Heavy Ion Workshop and CERN TH institute ``Novel tools and observables for jet physics in heavy-ion collisions'' took place at CERN 25 August -- 1 September 2017 \cite{THinst}. The main objective of the meeting was to bring together experimentalists and theorists in order to identify relevant jet observables that are sensitive to different aspects of the final-state interactions with the dense medium created in heavy-ion collisions and study the added potential of jet substructure techniques in this context.
%
%Due to the high level of complexity related to the experimental determination and theoretical treatment of jet observables, a comprehensive effort of comparing models against each other and against data on a wide range of observables is out of  scope and, probably, would not allow to draw meaningful conclusions. Instead, the workshop provided an opportunity to consider jet observables in heavy-ion collisions from a new perspective to a large extent made available through novel substructure techniques.
%This report sums up the main discussions that took place during the meeting, and presents a selected number of numerical studies using existing tools. These mainly include the Monte Carlo (MC) event generators PYTHIA 8 \cite{Sjostrand:2007gs} for simulating the proton-proton baseline, and in-medium jet evolution codes QPYTHIA \cite{Armesto:2009fj} and JEWEL \cite{Zapp:2011ya,Zapp:2012ak}. 
%%{\color{red} Some details about main assumptions and building blocks of JEWEL and Q-PYTHIA. Appendix?}
%The selected parton shower MC's were used for sake of convenience, and we look forward to extend the suite of studied generators, including MARTINI \cite{Schenke:2009gb,Young:2011ug}, HYDJET++ \cite{Lokhtin:2008xi}, YaJEM \cite{Renk:2010zx} and MATTER \cite{Majumder:2013re} as well as newly developed codes, most notably JETSCAPE \cite{Cao:2017zih}, in future editions of the workshop. In addition, for jet reconstruction we made extensive use of FastJet \cite{Cacciari:2005hq,Cacciari:2011ma} and for the purposes of additional pile-up mitigation in heavy-ion context, we studied constituent subtraction (CS) \cite{Berta:2014eza} and SoftKiller (SK) \cite{Cacciari:2014gra}.
In heavy ion collisions at LHC and at RHIC, essentially all hadronic high-$\pT$ observables deviate from baseline measurements in proton-proton collisions. The totality of these findings is generically referred to as ``jet-quenching''. It has been established first at RHIC on the level of single inclusive hadron spectra and high-$\pT$ hadron correlations. With the higher center of mass energies reached in nucleus-nucleus collisions at CERN's LHC, the focus has now moved to characterizing jet quenching in multi-hadron final states identified by modern jet finding algorithms.

There is overwhelming evidence that these jet quenching phenomena arise predominantly from final state interactions of the high-$\pT$ fragments of partonic scattering processes with the dense QCD matter produced in the collision region. Initial state effects such as nuclear modifications of parton distribution functions play a non-negligible but generally sub-dominant role. As a consequence, dynamical models of jet quenching focus mainly on the medium-induced modifications of final state parton showers. 

For the central aims of the relativistic heavy-ion programs at LHC and at RHIC, jet quenching is of interest for at least two reasons. First, experimental access to information about the dense QCD matter produced in heavy ion collisions can be obtained from the scattering of calibrated hard probes on that matter. 
In this respect, jet physics has been extensively constrained during decades of high-energy collider experiments and is a workhorse for studying perturbative and non-perturbative aspects of QCD \cite{Ellis:1991qj}.
Second, since high-$\pT$ fragments are far out-of-equilibrium, characterizing their medium-induced softening and isotropization may give access to the QCD equilibration processes and may thus help to understand whether, how and on which time scale heavy ion collisions evolve towards equilibrium. 

Jet quenching models, i.e., dynamical models of jet-medium interactions and the ensuing modifications of final state parton showers, are essential to 
infer from the measurements of quenched jets information about dense QCD matter and the equilibration processes that lead to it. Obviously, firm 
conclusions about QCD matter properties are only those that are independent of model-specific details, that are consistent with QCD theory and that are - within the range of validity of the dynamical model - consistent with the totality of experimental data. This necessitates the benchmarking of jet quenching models to establish where various models differ, and how experimental data can best discriminate between them.

With the aim of clarifying differences and similarities between various theoretical implementations of jet quenching on the level of single-hadron spectra, such a benchmarking exercise was organized by the TECHQM collaboration \cite{TECHQM}. For the success of this exercise, it was crucial to identify a simplified 
benchmark calculation that would allow for a comprehensive comparison since it could be easily preformed in all model set-ups. For the case of single inclusive
hadron spectra, this joint effort was the so-called ``brick problem'' \cite{Armesto:2011ht,Burke:2013yra}.

From 21 August -- 1 September 2017 \cite{THinst}, the CERN TH Institute  {\sl Novel tools and observables for jet physics in heavy-ion collisions} brought 
together at CERN 60 theorists and experimentalists with a research focus on jet quenching. The first three days of this two-week meeting were reserved 
to host the 5th in the series of Heavy Ion Jet Workshops.
% {\bf REF}. 
Already in the preparatory phase of these meetings, the question arose how a TECHQM-like community 
effort could be extended from single-inclusive hadron spectra to medium-modified multi-particle final states. Within the first days of the meeting, {\bf the 
participants reached the consensus view that a comprehensive comparison of existing and future jet quenching models can be based on the
distribution of hadronic fragments in the kinematical Lund plane.} 
This representation provides a common language to discuss features of final state showering on an operational level that provides a basis for comparing {\sl qualitative} and {\sl quantitative} features of theoretical models and physical observables.
The focus of the ensuing discussion was two-fold. On the one hand, there was an effort to understand
how and for which dynamical reasons different the fragment distributions of different jet quenching models show marked differences. Second, there was an
effort to understand how modern jet substructure techniques can be utilized to focus on particular regions in the Lund plane, thus providing a possibility of
constructing jet observables that discriminate optimally between different models. 

This report summarizes the main strategy as formulated during the CERN TH Institute, and the subsequent studies on which the participants agreed. 
The main body of numerical results shown in this report are based 
on  the Monte Carlo (MC) event generators PYTHIA 8 \cite{Sjostrand:2007gs} for simulating the proton-proton baseline, and on the 
codes QPYTHIA \cite{Armesto:2009fj} and JEWEL \cite{Zapp:2011ya,Zapp:2012ak} for simulating in-medium jet evolution.
  For jet reconstruction we made extensive use of FastJet \cite{Cacciari:2005hq,Cacciari:2011ma} and for the purposes of additional pile-up mitigation in heavy-ion context, we explored constituent subtraction (CS) \cite{Berta:2014eza} and SoftKiller (SK) \cite{Cacciari:2014gra}. It would have been of obvious 
interest to include in this comparison on equal par other existing jet quenching Monte Carlos, such as MARTINI \cite{Schenke:2009gb,Young:2011ug}, HYDJET++ \cite{Lokhtin:2008xi}, YaJEM \cite{Renk:2010zx} and MATTER \cite{Majumder:2013re}. However, in the light of the committements of the workshop participants 
to contribute to this write-up, and on the timescale foreseen for this first exercise, such a comprehensive comparison was not possible. 
The numerical studies included in this report are therefore illustrative for the general strategy of jet model comparison as formulated at the workshop, 
but they are not comprehensive since they are based only on a small number of simulations done with a subset of all existing tools. In the coming years,
we  foresee the further development of existing jet quenching models, and the advent of new ones (such as those envisaged in the framework of 
JETSCAPE \cite{Cao:2017zih}). The main aim of this write-up is to formulate a simple, generally applicable strategy for characterizing differences
between jet quenching models and devising observables that allow one to discriminate best between them. The report is organized as follows:

 In \autoref{sec:phasespace}, we introduce for the first time, in the context of heavy-ion studies, the concept of a splitting map, based on the kinematical Lund diagram \cite{Andersson:1988gp}. 
 %Filling t
 This map provides a representation of the radiation pattern implemented in Monte Carlo showering algorithms and allows to directly compare their features in different kinematical regimes. It also provides a direct, visual impression of what phase space region is being most significantly modified by medium effects. In detail, 
%We organize the report according to two main motifs. In the first part, \autoref{sec:phasespace}, we introduce for the first time, in the context of heavy-ion studies, the concept of a splitting map, based on the Lund kinematical diagram \cite{Andersson:1988gp}. Filling this map gives a operational representation of the radiation pattern implemented in a given showering algorithm. It also provides a direct, visual impression of what phase space region is being most significantly modified by medium effects. In detail, 
\begin{description}

\item[\autoref{sec:phasespace-theory}] gives a brief introduction to theoretical concepts that are useful for understanding the Lund diagram on the level of a single splitting, both for vacuum showers and showers in the medium.

\item[\autoref{sec:iterative-declustering}] describes in detail the procedure to fill the splitting map, by describing the steps related to jet reclustering and calculation of the variables that go into the map. 
%In particular, we study the impact of changing the reclustering algorithm, giving rise to different jet ``histories'' on the level of the PYTHIA vacuum shower. 

\item[\autoref{sec:phasespace-mc}] presents a study of the splitting maps of in-medium MC parton showers, QPYTHIA and JEWEL. 
We refer the interested reader to Appendix~\ref{app:models} for further details on the MC's utilized in the studies presented below. 
Finally, in \autoref{sec:uncorrelatedbackground}, we study the resilience of the observed   features at generator level to uncorrelated background by embedding the jet samples into a realistic heavy-ion environment.

\end{description}
While the splitting map contains the maximal amount of information, since it convolutes the kinematics of every splitting, it is also amenable to more selective examinations, for instance, through the implementation of jet ``tagging'' and ``grooming'' procedures. These tools are extensively used in the high-energy community \cite{Bendavid:2018nar} for a wide range of purposes, spanning jet substructure studies and leveraging this control for studies of observables beyond pure QCD, see e.g. \cite{Larkoski:2017jix,Asquith:2018igt} for the most recent reviews. They have also been previously applied in Monte-Carlo studies for heavy-ion collisions \cite{Zhang:2015trf,Apolinario:2017qay,Milhano:2017nzm}. 
In studying concrete observables, we have mainly focussed on applying the so-called Soft Drop (SD) grooming procedure with $\beta = 0$ \cite{Dasgupta:2013ihk} and $\beta \neq 0$ \cite{Larkoski:2014wba}, to be detailed below, that aims at identifying the first hard jet branching. 
Hence, in the second part of the report, \autoref{sec:jetsubstructure}, we perform a set of MC studies, on generator level and including embedding into a realistic heavy-ion background, using state-of-the-art grooming techniques. These observables are  not limited to substructure but are also used in order to extract more differential aspects from inclusive jet observables. In detail, 
\begin{description}
\item[\autoref{sec:groomedobservables}] presents the result for the groomed momentum-fraction $z_g$, subjet angle $\Delta R_{12}$ and the groomed mass  $M_g$ for QPYTHIA and JEWEL using three grooming settings. In this section, the studies were performed without embedding in a realistic background. We shed more light on the robustness of these results by checking the size of hadronization effects in the $z_g$ distribution for three different SD settings \autoref{sec:hadronization}. 
%As in \autoref{sec:iterative-declustering}, we also perform a brief study on how the reclustering algorithms modify the distributions.
\item[\autoref{sec:dissecting}] suggests a complementary look on substructure by submitting the jet sample that goes in to constructing a fully inclusive observable to an additional grooming step. Concretely, we propose to bin the inclusive jet sample in terms of the angular separation of the hardest subjets, $\Delta R_{12}$, using SD grooming. We demonstrate this procedure on QPYTHIA and JEWEL samples for the nuclear modification factor $R_{AA}$ and the photon-jet momentum imbalance in terms of the variable $x_{J\gamma}$. For these observables we also took into account embedding into a heavy-ion background \cite{deBarros:2012ws}.
\end{description}
Finally, we wrap up with an outlook in \autoref{sec:outlook}.



%Calculations of jet substructure \cite{Chien:2016led,Mehtar-Tani:2016aco,Milhano:2017nzm,Chang:2017gkt}, and data \cite{Sirunyan:2017bsd}.