%!TEX root = THinstituteReport_1.tex

%%%%%%%%%%%%%%%%%%%%%%%%%%%%%%%%%%%%%%%%%
\section{Jet substructure}
\label{sec:jetsubstructure}
%%%%%%%%%%%%%%%%%%%%%%%%%%%%%%%%%%%%%%%%%

As mentioned before, jet substructure techniques usually involve a step which reorganizes the constituents of a jet into a hierarchical tree where the nodes represent the splittings of sub-jets. 
This structure serves for subsequent analysis using additional techniques called jet grooming and tagging algorithms.
Grooming techniques usually reorganizes the tree by discarding radiation that fail to pass given criteria, typically soft and large-angle radiation. Taggers, on the other hand, aim at identifying the first splitting that passes a given criterion. In this way it splits a jet into two sub-jets. There has been a lot of progress recently utilizing these techniques for a wide range of substructure observables \cite{Butterworth:2008iy,Ellis:2009me,Krohn:2009th,Dasgupta:2013ihk,Larkoski:2014wba}, for a recent review see e.g. \cite{Larkoski:2017jix}.

While jet fragmentation functions and other jet shape observables have been studied experimentally until recently these techniques had not been used in the context of heavy-ion collisions.
Jet grooming was recently introduced as a tool to study the medium modification of leading partonic components in a parton shower~\cite{Sirunyan:2017bsd}, for related theoretical interpretations see \cite{Chien:2016led,Mehtar-Tani:2016aco,Milhano:2017nzm,Chang:2017gkt}. 
%Jet grooming algorithms are used to split a single jet into two subjets, a process referred to as ``declustering''~. 
These subjets provide access to the properties of the first splitting of the parton evolution in the vacuum~\cite{Altarelli:1977zs,Larkoski:2015lea}.

Given the proliferation of existing techniques, we will only refer to these as grooming techniques and, in our studies, concretely study one, namely the Soft Drop procedure.
The Soft Drop algorithm reclusters the anti-$k_{\mathrm{T}}$ jet constituents using C/A to create an angular-ordered clustering tree. On this tree a pairwise declustering is performed. In each step of the declustering the softer branch is removed until a branch is found that satisfies
%\begin{linenomath}
\beq
\label{eq:groompar}
\frac{\mathrm{min}(p_{\mathrm{T},i},p_{\mathrm{T},j})}{p_{\mathrm{T},i}+p_{\mathrm{T},j}} > z_{\text{cut}}\left( \frac{\Delta R_{ij}}{R_{0}} \right)^{\beta},
\eeq
%\end{linenomath}
where the subscripts ``$i$'' and ``$j$'' indicate the subjets at that step of the declustering, $\Delta R_{ij}$ is the distance between the two subjets, $R_{0}$ is the cone size of the anti-$k_{\mathrm{T}}$ jet, and $z_{\text{cut}}$ and $\beta$ are adjustable parameters. By varying $z_{\text{cut}}$ and $\beta$ specific regions of the emission phase space, see \autoref{fig:PS1}, can be isolated. For $\beta = 0$, this procedure is identical to the modified mass-drop tagger \cite{Dasgupta:2013ihk}. It allows to design specific grooming settings sensitive for example semi-hard radiation from single hard scatterings, soft radiation in the BDMPS regime and soft contribution originating from heating up the medium while the parton shower traverses it.

%%%%%%%%%%%%%%%%%%%%%%%%%%%%%%%%%%%%%
\begin{figure}[t!]
\centering
\includegraphics[width=0.33\textwidth]{figures/kinematics/plotVac_SD1_2}%
\includegraphics[width=0.33\textwidth]{figures/kinematics/plotVac_SD2_2}%
\includegraphics[width=0.33\textwidth]{figures/kinematics/plotVac_SD3_2}%
\caption{The three grooming settings studied in this report, see text for details, with the same medium scales as in \autoref{fig:PS1} (left). Shaded areas correspond to configuration that are groomed away.}
\label{fig:TheorySD}
\end{figure}
%%%%%%%%%%%%%%%%%%%%%%%%%%%%%%%%%%%%%
We compare three grooming settings:
\begin{description}
\item[{\bf SD1:}] $z_{\text{cut}}=0.1$ and $\beta=0$: this removes branches based on the momentum scale only;
\item[{\bf SD2:}]  $z_{\text{cut}}=0.5$ and $\beta=1.5$: this has a stronger grooming at large angle;
\item[{\bf SD3:}]  $z_{\text{cut}}=0.1$ and $\beta=-1.0$: this setting selects only the hard radiation;
\end{description}
How these different settings cut into the emission phase space is shown \autoref{fig:TheorySD}. While the first setting is the more widely used in various studies of the SD procedure, the two latter are designed to suppress regions of phase space with a lot of medium activity, as identified in the diagrams in \autoref{fig:PS2}. One could, of course, devise other grooming strategies, or even combine various conditions, in order to ``carve'' out kinematical regimes of particular interest. We avoid such prescriptions here in order not to bias our sample excessively. On the other hand, it could be interesting to combine grooming strategies with specific reclustering algorithms, a point we briefly study in \autoref{sec:hadronization}.
%{\color{red} some more description needed}

%\begin{figure}[h]
%\centering
%\includegraphics[width=0.33\textwidth]{figures/LundMC/Pythia_CA}
%\includegraphics[width=0.33\textwidth]{figures/LundMC/Pythia_NoUE}
%\caption{Lund diagram reconstructed from jets generated by PYTHIA8 with (left) and without underlying event (right)}
%\label{fig:PS2Vac}
%\end{figure}
%introduce grooming; explaining clustering/declustering
%%%%%%%%%%%%%%%%%%%%%%%%%%%%%%%%%%%%%%%%%%%%%
%\subsection{Radiation phase space and sensitivity to jet quenching {\color{green} Leticia,Harry}}
%\label{sec:radiationPSJQ}
%%%%%%%%%%%%%%%%%%%%%%%%%%%%%%%%%%%%%%%%%%%%%
%\subsubsection{Medium-induced radiation}
%%%%%%%%%%%%%%%%%%%%%%%%%%%%%%%%%%%%%%
%\begin{figure}[h]
%\centering
%\includegraphics[width=0.3\textwidth]{figures/LundMC/QPythiaHyd200}
%\includegraphics[width=0.3\textwidth]{figures/LundMC/JewelRecoilOff}
%\includegraphics[width=0.3\textwidth]{figures/LundMC/JewelRecoilOn}
%\includegraphics[width=0.3\textwidth]{figures/LundMC/QPythiaDiff200}
%\includegraphics[width=0.3\textwidth]{figures/LundMC/JewelRecoilOffDiff}
%\includegraphics[width=0.3\textwidth]{figures/LundMC/JewelRecoilOnDiff}
%\caption{Lund diagram reconstructed from jets generated by QPYTHIA (left), JEWEL without recoils (middle) and JEWEL with recoils on.
%The lower panels correspond to the difference of the radiation pattern between with and without jet quenching.}
%\label{fig:PS2}
%\end{figure}
%%%%%%%%%%%%%%%%%%%%%%%%%%%%%%%%%%%%%%
%As a demonstration of the general ideas outlined above, we have filled the Lund diagram using PYTHIA 8, and two pQCD-based models for jet quenching, namely JEWEL (w/wo medium recoils) and QPYTHIA.  The variables $z\theta$ and $\theta$ have been reconstructed from subsequent branchings that were identified by reclustering the jet with a C/A algorithm.
%
%Fig.~\ref{fig:PS2Vac} shows the Lund diagram in vacuum and its characteristic horizontal bands due to the evolution of the coupling constant with the momentum scale are apparent. The excess of splittings at large angle seen in the left plot is caused by the PYTHIA underlying event, which is switched off in the right plot. 
%
%Fig.~\ref{fig:PS2}, upper plots, correspond to QPYTHIA, JEWEL w/o recoils and JEWEL w/ recoils respectively. The lower plots show the differences to the corresponding vacuum Lund diagrams.
%
%QPYTHIA exhibits an modest excess of large $k_{T}$ quanta relative to vacuum. In the model, the number of splittings is increased relative to vacuum leading to a significant intra-jet momentum broadening. 
%
%JEWEL generates additional medium-induced branchings that are not present in the vacuum reference. These are also allowed to branch further in the medium. In the difference plot we can clearly identify a modest excess of large-angle, semi-hard quanta in JEWEL. Interestingly, jet profiles calculated with JEWEL do not show an enhancement of momentum at large angles relative to vacuum \cite{KunnawalkamElayavalli:2017hxo}. A possible reason is that in the case of the jet profiles the angle is always measured relative to the jet axis while in our declustering approach the angles are always measured relative to the hardest parent or subjet, in which case the angular distribution can be broader. 
%Such excess becomes harder when the medium recoils are included. The nature and role of the recoils will be explained in the next subsection. 
%
%It is worth noting that the medium-induced signal populates different regions of phase space in the two jet quenching models. The choice of the grooming parameters in Eq. \ref{eq:groompar} is critical to enhance the sensitivity to the signal, as it will be discussed in the next section. 
%
%
%\subsubsection{Medium response}
%
%Jet quenching is expected to be accompanied by recoil effects. The jet induced medium response constitutes a correlated background that can contribute to the modifications of the measured jet substructure. Recoil effects are expected to contribute in the soft-large angle sector of the phase space, similarly to the uncorrelated underlying event, described in the next section. 
%
%JEWEL can be ran with and without recoiling partons entering the hadronisation. 
%The difference between the Lund diagrams for the two options is shown in Fig.~\ref{fig:PS2} lower right plot. 
%
%The impact of the recoils as modeled by JEWEL has being documented \cite{Milhano:2017nzm}\cite{KunnawalkamElayavalli:2017hxo}. Its contribution is needed to describe most of the jet shapes measured so far at the LHC. In particular, if the medium response can smear the subleading subjet momentum above the given grooming cut, the subjet momentum balance or $z_{g}$ can become more asymmetric relative to vacuum.   
%
%As a correlated background, the medium response cannot be experimentally subtracted
%to isolate purely radiative modifications to the jet shower. However, correlation of jet substructure observables might help to suppress it \cite{Milhano:2017nzm}. 
%\begin{figure}[th]
%\centering
%\includegraphics[width=0.33\textwidth]{figures/LundMC/JewelRecoilDiff.pdf}%
%\caption{Difference of the lund plots for JEWEL with recoils on and without}
%\label{fig:RecoilJewel}
%\end{figure}

%\clearpage
%\newpage
%%%%%%%%%%%%%%%%%%%%%%%%%%%%%%%%%%%%%
\subsection{Groomed substructure observables and sensitivity to jet quenching}
\label{sec:groomedobservables}
%%%%%%%%%%%%%%%%%%%%%%%%%%%%%%%%%%%%%

After identifying the first splitting that satisfies Eq.~(\ref{eq:groompar}), we have access to the full kinematics of that branching process. The groomed jet-$\pT$ is now defined as $p_{{\rm \tiny T} g} \equiv p_{\mathrm{T},1}+p_{\mathrm{T},2}$, where the subscripts now refer to the chosen sub-jets. We can then define the groomed momentum fraction, $z_g = \min \left(p_{{\rm \tiny T},1},p_{{\rm \tiny T},2}\right)/p_{{\rm \tiny T}g}$ and the angle $\Delta R_{12}$. In our numerical studies, we will focus on these two quantities but also introduce the groomed mass $M_g$, defined as in Eq.~(\ref{eq:DipoleMass}) with all relevant quantities being groomed. These observables share light on how the branchings occur in course of the parton shower and are sensitive to medium effects as long as the branching originates from inside the medium, $ t_{{\rm f}g}\equiv 2 p_{{\rm \tiny T}g}/M_g^2 < L$. For the chosen medium parameters, the samples analyzed with settings SD1 and SD2 will contain an admixture of in-medium and out-of-medium splittings, see \autoref{fig:TheorySD}, while SD3 picks exclusively out hard splittings originating from inside the medium. 

As in the previous section, the MC's we use in our study are QPYTHIA and JEWEL (with recoil on and off) and are shown in \autoref{fig:SDGenZG}, \ref{fig:SDGenDR12} and \ref{fig:SDGenMg}. Jets were reconstructed using anti-$k_{\rm \tiny T}=0.4$ and have $\pT > 130$ GeV/c.
The results in this section are obtained from generator level. In particular, we have not introduced any detector resolution effects, such as a minimal angular cut-off $\Delta R_{\rm min}$.
Note, that the distributions are normalized by the total number of anti-$k_{\text{T}}$ (ungroomed) jets. The distributions are therefore not self-normalized and contain information how grooming affects the overall suppression of the (groomed) jet yield. 

%%%%%%%%%%%%%%%%%%%%%%%%%%%%%%%%%%%%%
\begin{figure}[t]
\centering
\includegraphics[width=0.33\textwidth]{figures/SDGen/ZgCompModelsBeta00Z01.pdf}%
\includegraphics[width=0.33\textwidth]{figures/SDGen/ZgCompModelsBeta15Z05.pdf}%
\includegraphics[width=0.33\textwidth]{figures/SDGen/ZgCompModelsBetam1Z01.pdf}%
\caption{Groomed shared momentum fraction, $z_{\mathrm{g}}$, for three different grooming settings in simulations with and without jet quenching. The uppers panels show the $z_{\mathrm{g}}$ distribution normalized by the total number of ungroomed jets while the lower panels show the ratio of JEWEL and QPYTHIA with respect to PYTHIA8.}
\label{fig:SDGenZG}
\end{figure}
%%%%%%%%%%%%%%%%%%%%%%%%%%%%%%%%%%%%%
Figure~\ref{fig:SDGenZG} shows the momentum fraction carried by the softest subjet for different event generators. The vacuum baseline is represented by the PYTHIA8 data points and compared to results from the QPYTHIA and JEWEL jet quenching event generators. The most striking feature is the generally opposite trend of the two models. This can also be traced back to the discussion around \autoref{fig:PS2}. The modified parton shower in QPYTHIA makes the jets broader with respect to jets in vacuum and therefore many more jets survive the grooming. JEWEL however collimates the jets and therefore less jets are surviving the grooming with this setting.

We also note, that while for $\beta \geq 0$, see \autoref{fig:SDGenZG} (left and center), the number of jets for the different generators remains roughly constant while for the negative grooming setting $\beta < 0$, \autoref{fig:SDGenZG} (right), a large deviation from unity can be observed. Interestingly, QPYTHIA subjets are strongly enhanced in this regime while JEWEL subjets are strongly suppressed, both by a factor $\sim2$.

Comparing the JEWEL results with and without recoil demonstrates that, for the chosen analysis settings, this observable is not very sensitive to recoil effects except for the the small-$z_g$ region. The tightest setting $\beta < 0$ is notably very resilient. In order to compare to the data presented in \cite{Sirunyan:2017bsd}, see also \cite{Milhano:2017nzm} for a study using JEWEL, where a significant deviation from vacuum baseline was observed, we again point out that no minimal angular cut-off was employed in our studies. Such a cut-off suppresses collinear vacuum radiation and, hence, amplifies the effects related to the medium.

%%%%%%%%%%%%%%%%%%%%%%%%%%%%%%%%%%%%%
\begin{figure}[th!]
\centering
\includegraphics[width=0.33\textwidth]{figures/SDGen/DR12CompModelsBeta00Z01.pdf}%
\includegraphics[width=0.33\textwidth]{figures/SDGen/DR12CompModelsBeta15Z05.pdf}%
\includegraphics[width=0.33\textwidth]{figures/SDGen/DR12CompModelsBetam1Z01.pdf}%
\caption{Distance between the two groomed subjets, $\Delta R_{12}$, for three different grooming settings in simulations with and without jet quenching. The uppers panels show the $\Delta R_{\mathrm{12}}$ distribution normalized by the total number of ungroomed jets while the lower panels show the ratio of JEWEL and QPYTHIA with respect to PYTHIA8.}
\label{fig:SDGenDR12}
\end{figure}
%%%%%%%%%%%%%%%%%%%%%%%%%%%%%%%%%%%%%
Next we  turn to studying the angular region where medium effects set in. One particularly interesting aspect is whether substructures are quenched according to their angular separation. The angular distance between the groomed sub-jets is plotted in \autoref{fig:SDGenDR12} for the three grooming settings. Once again, we see big differences between the MC models; JEWEL being very collimated and QPYTHIA very broad. In the JEWEL samples with $\beta \geq 0$ one observes larger sensitivity to recoil effects than for $\beta < 0$, which is remarkably flat for a wide range of angles.

Finally, we study the groomed jet mass normalized by the transverse momentum, $M_{\mathrm{g}}/p_{\mathrm{T,jet}}$, in \autoref{fig:SDGenMg}. This observable combines the features already seen before. We note, in particular, a strong resilience to recoil effects in the JEWEL samples for all SD settings. Due to the suppression of large-angle jet structures in JEWEL (collimation), the small $M_g$ region is significantly enhanced compared to the vacuum baseline. 
%{\color{red} An interesting feature is perhaps that JEWEL is modified for small $M_g$ which technically implies that the splitting takes place outside of the medium. This is of course related to the enhancement seen at very small $\Delta R_{12}$. What is the reason for this modification? QPYTHIA is not modified there.}
%%%%%%%%%%%%%%%%%%%%%%%%%%%%%%%%%%%%%
\begin{figure}[th]
\centering
\includegraphics[width=0.33\textwidth]{figures/SDGen/MgOverPtgCompModelsBeta00Z01.pdf}%
\includegraphics[width=0.33\textwidth]{figures/SDGen/MgOverPtgCompModelsBeta15Z05.pdf}%
\includegraphics[width=0.33\textwidth]{figures/SDGen/MgOverPtgCompModelsBetam1Z01.pdf}%
\caption{Groomed jet mass, $M_{\mathrm{g}}/p_{\mathrm{T,jet}}$, for three different grooming settings in simulations with and without jet quenching. The uppers panels show the $M_{\mathrm{g}}/p_{\mathrm{T,jet}}$ distribution normalized by the total number of ungroomed jets while the lower panels show the ratio of JEWEL and QPYTHIA with respect to PYTHIA8.}
\label{fig:SDGenMg}
\end{figure}
%%%%%%%%%%%%%%%%%%%%%%%%%%%%%%%%%%%%%
%

%%\clearpage
%%\newpage
%%%%%%%%%%%%%%%%%%%%%%%%%%%%%%%%%%%%%%%%%%
%\subsection{Sensitivity to reclustering algorithm} {\color{green} Leticia, Harry}
%\label{sec:reclusteringalgo}
%%%%%%%%%%%%%%%%%%%%%%%%%%%%%%%%%%%%%%%%%%
%The strategy of the clustering algorithm manifests itself in the declustering and grooming. CA algorithm combines the closest particles first. As a consequence, the first declustering steps will encounter soft splittings at large angle. In the case of k$_{T}$ algorithm, the softest particles are clustered first. As a consequence, the first declustering steps will encounter hard splittings. Anti-k$_{T}$ clusters hard particles first, thus splittings at the first declustering steps will be generally soft. Such ordering is reflected in the Lund plots of \autoref{fig:AlgoDependence}.
%\begin{figure}[th]
%\centering
%\includegraphics[width=0.33\textwidth]{figures/LundMC/Pythia_CA.pdf}%
%\includegraphics[width=0.33\textwidth]{figures/LundMC/Pythia_kt.pdf}%
%\includegraphics[width=0.33\textwidth]{figures/LundMC/Pythia_Akt.pdf}%
%\caption{Lund plot for different declustering strategies: CA (left), k$_{T}$(middle), Anti-k$_{T}$ (right)}
%\label{fig:AlgoDependence}
%\end{figure}
%
%
%In Fig. \ref{fig:AlgoDependenceSignal}, left plot, the difference between JEWEL (recoils off) and vacuum is shown for k$_{T}$ declustering strategy. The similar plot for QPYTHIA is shown on the right plot. We see that in the case of JEWEL, the excess of splittings at large angle is of the same order of magnitude with CA and k$_{T}$ strategies and around 20 $\%$. In the case of QPYTHIA, the excess of splittings at large $k_{T}$ is enhanced to 20$\%$ with k$_{T}$ strategy compared to the 8$\%$ enhancement we observe for CA strategy in Fig. \ref{fig:PS2Vac} lower left plot. 
%
%
%\begin{figure}[th]
%\centering
%\includegraphics[width=0.33\textwidth]{figures/LundMC/SignalGeneratorDiff_kt.pdf}%
%\includegraphics[width=0.33\textwidth]{figures/LundMC/QPythiaDiff_kt.pdf}%
%\caption{Difference of radiation pattern of JEWEL (no recoils) relative to vacuum using k$_{T}$ strategy (left). Same plot for QPYTHIA (right)}
%\label{fig:AlgoDependenceSignal}
%\end{figure}
%
%
%%%%%%%%%%%%%%%%%%%%%%%%%%%%%%%%%%%%%%%%%
\subsubsection{Sensitivity to hadronization and reclustering algorithm}
\label{sec:hadronization}
%%%%%%%%%%%%%%%%%%%%%%%%%%%%%%%%%%%%%%%%%


The last stage of the jet fragmentation is the non-perturbative process of hadronization. This is a dynamical process that converts colored partons into color-singlet hadrons. In jet quenching event generators it is assumed that hadronization occurs outside of the medium. A proof for this assumption does not exist and therefore hadronization uncertainties should be expected to be sizable. 

%%%%%%%%%%%%%%%%%%%%%%%%%%%%%%%%%%%%%
\begin{figure}[th]
\centering
\includegraphics[width=0.33\textwidth]{figures/SDGen/ZgPytHadVsPartBeta00Z01.pdf}%
\includegraphics[width=0.33\textwidth]{figures/SDGen/ZgPytHadVsPartBeta15Z05.pdf}%
\includegraphics[width=0.33\textwidth]{figures/SDGen/ZgPytHadVsPartBetam1Z01.pdf}%
\caption{Groomed shared momentum fraction, $z_{\mathrm{g}}$, for three different grooming settings in simulations with and without hadronization with the PYTHIA8 event generator.}
\label{fig:SDGenZGHadVsPart}
\end{figure}
%%%%%%%%%%%%%%%%%%%%%%%%%%%%%%%%%%%%%
Even for vacuum physics, it is well known that the SD procedure has some sensitivity to hadronization effects, for $\beta = 0$ see \cite{Dasgupta:2015yua}.
From perturbative arguments hadronization corrections to the jet $p_{T}$ grow like $R^{-1}$ \cite{Dasgupta:2007wa} and
%inversely proportional to the jet resolution $R$: $\Delta p_{T}^{had} \approx \frac{-1}{R}$, 
so are potentially important for subjet observables.
However, since hadronization is a process that happens locally in phase space, jets are less sensitive to the hadronization uncertainties than observables based on hadrons. In this paragraph we investigate how sensitive groomed subjet observables are to the hadronization process. For this purpose we compare the \zg-distribution in PYTHIA8 with and without hadronization for the three SD settings, described above, as shown in \autoref{fig:SDGenZGHadVsPart}. It can be observed that the low-\zg\, region is particularly sensitive to hadronization effects. For grooming with negative $\beta$ the hadron- and parton-level results are most similar, see Fig,~\ref{fig:SDGenZGHadVsPart} (right), because with these grooming settings the soft splittings are rejected. Dedicated studies of these effects in conjunction with medium-modified hadronization are left for the future.


%%%%%%%%%%%%%%%%%%%%%%%%%%%%%%%%%%%%%
\begin{figure}[th]
\centering
\includegraphics[width=0.32\textwidth]
{figures/SDAlgorithms/zgClusteringComp.pdf}
\includegraphics[width=0.32\textwidth]
{figures/SDAlgorithms/rgClusteringComp.pdf}
\includegraphics[width=0.32\textwidth]
{figures/SDAlgorithms/mgClusteringComp.pdf}%
\caption{Subset of grooming variables, symmetry parameter ($z_{g}$), groomed mass ($M_{g}$) and groomed radius ($\Delta R_{12}$) for three different jet reclustering algorithms.}
\label{fig:SDClusteringComp}
\end{figure}
%%%%%%%%%%%%%%%%%%%%%%%%%%%%%%%%%%%%%
Finally, we studied the behavior of the three observables subject to different reclustering algorithms applied, see \autoref{fig:SDClusteringComp}. In this particular case, we limit ourselves only to looking at the PYTHIA samples.
In case of a grooming prescription that requires a semi-hard splitting, for instance like in the SD1 setting,
%$z_{cut}$=0.1, $\beta=0$, 
the number of groomed branches will be large for anti-$k_{\rm \tiny T}$ reclustering ($\lesssim 30$) and very small for $k_{\rm \tiny T}$, for which the grooming conditions will be satisfied at the first iteration in most of the cases. Consistently, the groomed momentum fraction \zg\, probes very asymmetric splittings in the case of anti-$k_{\rm \tiny T}$ reclustering as can be seen in \autoref{fig:SDClusteringComp} (left). In contrast, $k_{\rm \tiny T}$-reclustered \zg\, picks exclusively up symmetric splittings, resulting in an almost featureless distribution. Similar conclusions can be made for the $\Delta R_{12}$ distribution, \autoref{fig:SDClusteringComp} (center), and $M_g$, \autoref{fig:SDClusteringComp} (right), as well.

%We postpone similar studies, that additionally include medium effects, to the future.


%%%%%%%%%%%%%%%%%%%%%%%%%%%%%%%%%%%%%%%%
\subsection{Unrolling/dissecting jet quenching observables using grooming}
\label{sec:dissecting}
%%%%%%%%%%%%%%%%%%%%%%%%%%%%%%%%%%%%%%%%

Many jet quenching observables, such as the nuclear modification factor $R_{AA}$ and the momentum imbalance in photon-jet events, are considered benchmark measurements. However, their constraining power to discriminate between models have also been questioned. In some cases, the influence of background fluctuations can also obscure their constraining power.

In this section we present studies of conventional jet quenching observables that are enhanced by ``unrolling'' the jet samples using grooming techniques. As a first step, we apply SD grooming on the jet sample, extracting from each on the grooming variables $z_g$ and $\Delta R_{12}$. From these variables we can divide the sample in many ways. We have simply reorganized the fully inclusive sample according to the angle separating the two hardest prongs of a particular jet. This is motivated by the splitting maps and the results obtained for the substructure observables previously. Another motivation is to differentiate between the modifications of the ``soft'' and the ``hard'' structure of the jet. The former is more dominant for inclusive observables and for non-restrictive SD settings, e.g. SD1 and SD2 in \autoref{fig:TheorySD} (left and central panels), while the latter would be more pronounced for conservative SD parameter choices, such as SD3 in \autoref{fig:TheorySD} (right panel).

We have not attempted to study this in any systematic way. Here, we only report on two sample studies at LHC of the $R_{AA}$ unrolled with SD1 and the $x_{J\gamma}$ distribution unrolled with SD2. More importantly, all results in this section have been computed by embedding the MC jet samples into a realistic heavy-ion background that depends on centrality. Hence they represent more realistically the magnitude of effects that should be expected to arise in heavy-ion collisions at the LHC.
%{\color{red} Are the curves for $\Delta R>0.3$ consistent with the expected effect of uncorrelated background at large angles...? Was additional pile-up mitigation employed?}


%%%%%%%%%%%%%%%%%%%%%%%%%%%%%%%%%%%%%
\begin{figure}[th]
\centering
\includegraphics[width=0.32\textwidth]{figures/Observables_RAA/Plot9}
\includegraphics[width=0.32\textwidth]{figures/Observables_RAA/Plot3}
\includegraphics[width=0.32\textwidth]{figures/Observables_RAA/Plot4}
%\includegraphics[width=0.33\textwidth]{figures/LundMC/PythiaDiffCA_wLines200.pdf}%
%\includegraphics[width=0.33\textwidth]{figures/LundMC/PythiaDiffCA_wLines80120.pdf}%
\caption{The nuclear modification factor for subsamples of jets that have been unrolled as a function of $\Delta R$ of the leading sub-jets identified using SD1. 
}
\label{fig:GroomedRAA}
\end{figure}
%%%%%%%%%%%%%%%%%%%%%%%%%%%%%%%%%%%%%
The well-known nuclear modification factor, $R_{AA}$, is a standard benchmark for estimating/tuning medium parameters in jet quenching calculations. However, by dividing the sample of inclusive high-$\pT$ jets into small- and large-angle configurations we gain access to more differential information regarding the accompanying modifications of the intra-jet structure. Similar studies, albeit using another method to dissect the jet sample into two-prong structures, was already presented in \cite{Zhang:2015trf,Apolinario:2017qay}.

The jet samples generated from QPYTHIA, JEWEL ``Recoil off'' and JEWEL ``Recoil on'' that goes into calculating $R_{AA}$ in \autoref{fig:GroomedRAA}, has been divided using SD3 grooming into samples related to the angular separation of the hardest branches. While all three models gives a similar $\pT$-trend of $R_{AA}$ for the fully inclusive sample (see black points in \autoref{fig:GroomedRAA}), large differences are seen for the unrolled results.\footnote{The overall magnitude of the inclusive $R_{AA}$ does not play an important role for the point we are trying to make.} In QPYTHIA, the core of the jet is quenched stronger than the periphery, as expected from previous studies above. For JEWEL(recoil-off), the effect is completely opposite: the jet core is quenched much less than large-angle splittings. This comes as no surprise in light  of other substructure observables that were analyzed above, see e.g. \autoref{sec:groomedobservables}. Including recoil effects, the JEWEL sample contains a strong $\pT$-dependence of large-angle jets and stands completely out. This could hint of an enhanced sensitivity to medium recoil in this observable.

%%%%%%%%%%%%%%%%%%%%%%%%%%%%%%%%%%%%%
\begin{figure}[th]
\centering
%\includegraphics[width=0.5\textwidth]
%{figures/Observables_GammaJet/GammaJet_groomed}%
\includegraphics[width=0.5\textwidth]
{figures/Observables_GammaJet/JEWEL-photon-jet-recoilOn-linear}%
\caption{The $x_{J\gamma}$ distribution for subsamples of jets that have been unrolled as a function of the angle found between the leading sub-jets using SD. }
\label{fig:GroomedGammaJet}
\end{figure}
%%%%%%%%%%%%%%%%%%%%%%%%%%%%%%%%%%%%%
Another benchmark observable is the photon-jet momentum asymmetry. We recall that the variable $x_{J\gamma}$ is defined as the ration of jet to photon momentum, $x_{J\gamma} \equiv p_{\perp,\text{jet}}/p_{\perp,\gamma}$.
In \autoref{fig:GroomedGammaJet} we have only included results for JEWEL with recoils turned on. Again, this sample has been unrolled as described above, this time using SD2 grooming. The same features that have been pointed out multiple times, also show up here as a function of collision centrality. Notable, the small-angle sample shows very little dependence of centrality, and is closely peaked around 1. The large-angle sample, on the other hand, is already very different in vacuum (i.e. the 90-100\% curve in \autoref{fig:GroomedGammaJet}) and evolves significantly from central to peripheral. 

These proof-of-point studies illustrate the enhanced sensitivity to more than one variable one obtains by unrolling the underlying jet sample using a well-controlled procedure.
However, the results shown in this section are only exploratory and more systematic studies are left for the future.
