%!TEX root = THinstituteReport_1.tex

%%%%%%%%%%%%%%%%%%%%%%%%%%%%%%%%%%%%%%%%
\section{Outlook}
\label{sec:outlook}
%%%%%%%%%%%%%%%%%%%%%%%%%%%%%%%%%%%%%%%%

The investigation of QCD jet observables in heavy-ion collisions is a community-wise effort, involving both experimentalists and theorists. 
%While s
Significant progress, both from the point of view of the development of experimental techniques as well as from theoretically motivated parametric estimates grounded on scale analysis and modeling within Monte Carlo parton showers, has led to a detailed qualitative understanding of how jets are modified in the medium created in the aftermath of heavy-ion collisions.
%, the field has not reached the level of precision associated with jet measurements in other colliding systems, such as proton-proton and DIS.
It is therefore worthwhile considering 
%whether it be possible and fruitful to contemplate 
strategies that would be useful to further enhance jet observables as unique and valuable probes of the quark-gluon plasma.
%consolidate measurements with first-principle theory in order to establish well-controlled baselines across jet samples and observables. 
As a first attempt at such a proposal, 
\emph{it is the consensus view of the participants of this workshop, that mapping the jet dynamics onto the kinematical Lund plane allows for a comprehensive comparison of jet quenching models.
This representation, which also could be useful for presenting experimental data, allows to pin down kinematical regimes where modifications arise and devise observables that are particularly constraining between existing and future models.}
%ambitious step would be to find a common language within the field of heavy-ions for comparisons between experimental data and theory. 
%However, it is almost as important to develop common ground with the wider field of high-energy physics, based on the language of perturbative QCD and the tools of modern high-energy experiments.
It is as important to develop common synergies within the wide field of jet physics at colliders, based on modern perturbative QCD (fixed-order and resummation) techniques and on improved tools for high-energy experiments.
%\kmt{Suggestions for future studies.}

The organization of the 5th Heavy Ion Jet Workshop and CERN Theory Institute {\sl ``Novel tools and observables for jet physics in heavy-ion collisions''} provided an opportunity to initiate a first, community-wise effort in this direction.
%work toward this goal.
%comparisons between various implementations of jet quenching modeling. The Lund kinematical diagram provides an eagle's view of the branching process as implemented in a particular code, avoiding the model details that often distract from seeing the big picture.
While the concrete calculations and model studies presented in this report reflect the discussions that took place at that moment, the main messages are however relevant for the field at large. Let us summarize them in two points.
\begin{itemize}

\item We have introduced an operational way to map the full content of a jet splitting process, making use of the Lund diagram. Using kinematical arguments, we can make sense of enriched and depleted regions of phase space as results of medium interactions and recoil. An important caveat is that this idealized picture gets strongly distorted due to the presence of uncorrelated background but we have shown, through various exercises, that this aspect mostly affects the low-$\pT$ observables.

\item We have outlined a strategy, based on a selection of variables after jet reclustering/grooming, to single out jet samples enriched in configurations possessing specific properties as an aid to single out physics mechanisms, in particular \textsl{hard} (e.g. medium-induced bremsstrahlung, modifications of intra-jet structure due to energy loss) from \textsl{soft} (e.g. particle yield, sensitivity to recoil) medium effects and, similarly, \textsl{large-angle} from \textsl{small-angle} components.

\end{itemize}
We hope the topics we have reported here will trigger new and exciting future studies of jet, and in particular jet substructure, observables in heavy-ion collisions.

